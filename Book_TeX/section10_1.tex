\subsection{Порядок выполнения работы}

\begin{enumerate}
    \itemВам необходимо реализовать один (для оценки <<3>>) или два (для оценки <<4>>) этапа вашей программы из предыдущих лабораторных работ. При этом вычисления можно проводить как на CPU, так и на GPU (на своё усмотрение, но GPU предпочтительнее).
    \item\textbf{Необязательное задание №1 (для получения оценки <<5>>).}
        \begin{itemize}
            \itemВыполнение заданий для оценки <<3>> и <<4>>.
            \itemРасчёт доверительного интервала. 
            \itemПосчитать время двумя способами: с помощью profiling и с помощью обычного замера (как в предыдущих заданиях).
            \itemОценить накладные расходы, такие как доля времени, проводимого на каждом этапе вычисления (<<нормированная диаграмма с областями и накоплением>>), число строк кода, добавленных при распараллеливании, а также грубая оценка времени, потраченного на распараллеливание (накладные расходы программиста), и т.п.
        \end{itemize}
    \item\textbf{Необязательное задание №2 (для получения бонусов и лучшей итоговой оценки по итогам прохождения дисциплины).} Провести вычисления совместно на GPU и CPU (т.е. итерации в некоторой обоснованной пропорции делятся между GPU и CPU, и параллельно на них выполняются).
    \itemПри желании данную лабораторную работу можно написать на CUDA.
\end{enumerate}

