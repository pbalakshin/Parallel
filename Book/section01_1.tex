\subsection{История развития параллельных вычислений}
\label{subsec:parallel-history}

Разговор о развитии параллельного программирования принято начинать истории развития суперкомпьютеров.
Однако первый в мире суперкомпьютер CDC6600, созданный в 1963 г., имел только один центральный процессор, поэтому едва ли можно считать его полноценной SMP-системой.
Архитектура SMP (от англ.\ Symmetric Multiprocessing) подразумевает работу несколько идентичных процессоров с общей для них оперативной памятью.
Многоядерный процессор можно считать частным случаем SMP-системы.

Третий в истории суперкомпьютер CDC8600 проектировался для использования четырёх процессоров с общей памятью, что позволяет говорить о первом случае применения SMP, однако CDC8600 так никогда и не был выпущен: его разработка была прекращена в 1972 году.

Лишь в 1983 году удалось создать работающий суперкомпьютер (Cray X-MP), в котором использовалось два центральных процессора, использовавших общую память.
Справедливости ради стоит отметить, что чуть раньше (в 1980 году) появился первый отечественный многопроцессорный компьютер Эльбрус-1, однако он по производительности значительно уступал суперкомпьютерам того времени.

Уже в 1994 можно был свободно купить настольный компьютер с двумя процессорами, когда компания ASUS выпустила свою первую материнскую плату с двумя сокетами, т.е. разъёмами для установки процессоров.

Следующей вехой в развитии SMP-систем стало появление многоядерных процессоров.
Первым многоядерным процессором массового использования стал POWER4, выпущенный фирмой IBM в 2001 году.
Но по-настоящему широкое распространение многоядерная архитектура получала лишь в 2005 году, когда компании AMD и Intel выпустили свои первые двухъядерные процессоры.

На рисунке~\ref{fig:GraphPartOfMultiCoreProcessorFromYear} показано, какую долю занимали процессоры с разным количеством ядер при создании суперкомпьютеров в разное время~\cite{Top500}. %(по материалам сайта \url{http://top500.org}).
Закрашенные области помечены цифрами для обозначения количества ядер.
Ширина области по вертикали равна относительной частоте использования процессоров соответствующего типа в рассматриваемом году.

\begin{figure}[H]
    \centering
    \includegraphics[width=0.95\linewidth]{GraphPartOfMultiCoreProcessorFromYear}
    \caption{Частотность использования процессоров с различным числом ядер при создании суперкомпьютеров}
    \label{fig:GraphPartOfMultiCoreProcessorFromYear}
\end{figure}

Как видим, активное использование двухъядерных процессоров в суперкомпьютерах началось уже в 2002 году, а примерно к 2005 году совершенно сошло на нет, тогда как в настольных компьютерах их применение в 2005 году лишь начиналось.
На основании этого можно сделать простой прогноз распространённости многоядерных <<настольных>> процессоров к нужному году, если считать, что они в общих чертах повторяют развитие многоядерных архитектур суперкомпьютеров.
