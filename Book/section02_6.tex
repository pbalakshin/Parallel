\subsection{Профилирование параллельных программ}
\label{subsec:profiling-parallel-programs}

Профилирование -- сбор характеристик работы программы, таких как время выполнения отдельных фрагментов (обычно подпрограмм), число верно предсказанных условных переходов, число кэш-промахов и т. д. Инструмент, используемый для анализа работы, называют профилировщиком или профайлером.

\textbf{Intel Parallel Amplifier.} Этот инструмент позволяет найти те участки кода, которые наиболее часто исполняются на процессоре. Также он позволяет оценить масштабируемость вашего параллельного приложения. И если есть какие-то проблемы с масштабируемостью, то найти те участки кода, которые этой масштабируемости мешают. В Intel Parallel Amplifier представлено три вида анализа:

\begin{enumerate}
    \item Hotspot-анализ -- Позволяет узнать где тратятся вычислительные ресурсы, а также изучить стек вызовов.
    \item Concurrency-анализ -- Происходит оценка эффективности параллельного кода.
    \item Lock\&Wait-анализ -- Указывает на те места, где программа плохо распараллеливается.
\end{enumerate}

Пройдя все эти этапы анализа, пользователь должен сформировать для себя определенное понимание поведения приложения в плане загрузки микропроцессора и эффективного использования его ресурсов. Далее на основе полученных результатов, можно решать дальнейшие шаги оптимизации программы.

Больше информации о Intel Parallel Amplifier можно посмотреть в~\cite{IntelAmplifier}.
% \url{https://www.ixbt.com/soft/intel-parallel-amplifier.shtml}.
