\thispagestyle{empty}
 
Соснин~В.В., Балакшин~П.В., Мишенёв~А.В.
Введение в параллельные вычисления.
--- СПб: Университет ИТМО, \the\year.
--~\pageref{LastPage}~с. \\

Рецензент: Зилинберг А.Ю., к.т.н., доцент кафедры радиотехнических систем, ФГАОУ ВО <<Санкт-Петербургский государственный университет аэрокосмического приборостроения>>. \\

В пособии излагаются основные понятия и определения теории параллельных вычислений.
Рассматриваются основные принципы построения программ на языке <<Си>> для многоядерных и многопроцессорных вычислительных комплексов с общей памятью.
Предлагается набор заданий для проведения лабораторных и практических занятий. \\

Учебное пособие предназначено для студентов, обучающихся по магистерским программам направления <<09.04.04 -- Программная инженерия>>, <<09.04.01 -- Информатика и вычислительная техника>>, и может быть использовано выпускниками (бакалаврами и магистрантами) при написании выпускных квалификационных работ, связанных с проектированием и исследованием многоядерных и многопроцессорных вычислительных комплексов.

\vspace*{\fill}

\begin{flushright}
    \includegraphics[scale=0.15]{itmo_logo_black_2022}
\end{flushright}

Университет ИТМО -- ведущий вуз России в области информационных и фотонных технологий, один из немногих российских вузов, получивших в 2009 году статус национального исследовательского университета.
С 2013 года Университет ИТМО -- участник программы повышения конкурентоспособности российских университетов среди ведущих мировых научно-образовательных центров, известной как проект <<5 в 100>>.
Цель Университета ИТМО -- становление исследовательского университета мирового уровня, предпринимательского по типу, ориентированного на интернационализацию всех направлений деятельности.

\begin{flushright}
    \copyright\spaceУниверситет ИТМО, \the\year
    
    \copyright\spaceСоснин~В.В., Балакшин~П.В., Мишенёв~А.В., \the\year
\end{flushright}
