\subsection{Порядок выполнения работы}
\begin{enumerate}
    \itemВзять в качестве исходной OpenMP-программу из ЛР №4, в которой распараллелены все этапы вычисления. Убедиться, что в этой программе корректно реализован одновременный доступ к общей переменной, используемой для вывода в консоль процента завершения программы.
    \itemИзменить исходную программу так, чтобы вместо OpenMP-ди\-рек\-тив применялся стандарт <<POSIX Threads>>:
        \begin{itemize}
            \item\textbf{для получения оценки <<3>>} достаточно изменить только один этап (Generate, Map, Merge, Sort), который является узким местом (bottleneck), а также функцию вывода в консоль процента завершения программы;
            \item\textbf{для получения оценки <<4>> и <<5>>} необходимо изменить всю программу, но допускается в качестве расписания циклов использовать <<schedule static>>;
            \item\textbf{для получения оценки <<5>>} необходимо хотя бы один цикл распараллелить, реализовав вручную расписание <<schedule dynamic>> или <<schedule guided>>.
        \end{itemize}
    \itemПровести эксперименты и по результатам выполнить сравнение работы двух параллельных программ (<<OpenMP>> и <<POSIX Threads>>), которое должно описывать следующие аспекты работы обеих программ (для различных $N$):
        \begin{itemize}
            \itemполное время решения задачи;
            \itemпараллельное ускорение;
            \itemдоля времени, проводимого на каждом этапе вычисления (<<нормированная
диаграмма с областями и накоплением>>);
            \itemколичество строк кода, добавленных при распараллеливании, а также грубая оценка
времени, потраченного на распараллеливание (накладные расходы программиста);
            \itemостальные аспекты, которые вы выяснили самостоятельно \textbf{(обязательный пункт)}.
        \end{itemize}
\end{enumerate}
