\phantomsection
\section*{Заключение}
\addcontentsline{toc}{section}{Заключение}

В данном пособии авторами рассматриваются основные аспекты параллельных вычислений. Пособие включает в себя две взаимосвязанные части: первая (разделы 1-4) посвящена теоретической подготовке обучающихся, вторая (разделы 5-10) связана с практической подготовкой выполнения лабораторных работ на базе первой части в области параллельного программирования. Акцент делается на преодолении порога вхождения в парадигму параллельных вычислений, на формировании навыков работы с рядом современных компиляторов и умений определения теоретических и практических значений оценки работы параллельных программ. Для дальнейшего изучения авторы советуют обратить внимание на инструментальные программные средства, предназначенные для отладки использования памяти (например, Valgrind), интерфейс обмена данными MPI, программный интерфейс Vulkan и распределённые вычисления.

Авторы выражают благодарность Балакшину Д.В., Косякову М.С., Шинкаруку Д.Н., Тараканову Д.С., Перминову И.В., Томчуку К.К., а также студентам и выпускникам факультета ПИиКТ Пначину И.Л., Томилову Н.А., Румянцевой М.Ю., Губареву В.Ю., Мирскому О.В. и другим за помощь в составлении и корректировке заданий.

Замечания, пожелания, комментарии направлять по адресу: pvbalakshin AT itmo.ru. 
