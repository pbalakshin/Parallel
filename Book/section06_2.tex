\subsection{Состав отчета}
\begin{enumerate}
    \itemТитульный лист с названием вуза, ФИО студентов и названием работы.
    \itemСодержание отчета (с указанием номера страниц и т.п.).
    \itemКраткая характеристика использованного для проведения экспериментов процессора, операционной системы и компилятора (официальное название, номер версии/модели, разрядность, число ядер, ёмкость ОЗУ, размер кэша и т.п.).
    \itemОписание особенностей конфигурации использованной параллельной библиотеки, включая описание последовательности шагов, предпринятых для установки библиотеки, и использованных опций компиляции.
    \itemПолный текст полученной параллельной программы, а также текст всех скриптов, использованных для компилирования программы и проведения экспериментов.
    \itemГрафики функций времени выполнения использованных программ, а также графики параллельного ускорения и параллельной эффективности для разных N и M (допускается совмещать несколько графиков в одной системе координат). Предпочтительно использовать столбчатые гистограммы.
    \itemПодробные выводы с анализом приведённых графиков и полученных результатов.
    \itemОтчёт предоставляется в бумажном или электронном виде вместе с полным текстом программы. По требованию преподавателя нужно быть готовыми скомпилировать и запустить этот файл на компьютере в учебной аудитории (или своём ноутбуке).
\end{enumerate}
