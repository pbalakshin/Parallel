\subsection{Состав отчета}
\begin{enumerate}
    \itemТитульный лист с названием вуза, ФИО студентов и названием работы. Содержание отчета (с указанием номера страниц и т.п.).
    \itemКраткое описание решаемой задачи.
    \itemХарактеристика использованного для проведения экспериментов процессора, операционной системы и компилятора GCC (точное название, номер версии/модели, разрядность, число ядер и т.п.).
    \itemПолный текст распараллеленной программы (для п.~2 и п.~3).
    \itemПодробные выводы.
    \itemОтчёт предоставляется в бумажном или электронном виде вместе с полным текстом программы. По требованию преподавателя нужно быть готовыми скомпилировать и запустить этот файл на компьютере в учебной аудитории (или своём ноутбуке).
\end{enumerate}

