\subsection{Библиотека Intel IPP}
\label{IPP:section}

\textbf{Оптимизация типовых задач обработки данных.} Существует немногочисленное количество высокопроизводительных библиотек, состоящих из набора низкоуровневых API для обработки данных: изображений, сигналов, матриц.

Одной из таких библиотек является <<Intel IPP>>\footnote{Intel Integrated Performance Primitives}, реализующая следующие функции:
\begin{itemize}
    \itemКодирование и декодирование видео.
    \itemКодирование и декодирование аудио.
    \itemКомпьютерное зрение.
    \itemКриптография.
    \itemСжатие данных.
    \itemПреобразование цвета.
    \itemОбработка изображения.
    \itemТрассировка луча/визуализация.
    \itemОбработка сигналов.
    \itemКодирование речи.
    \itemРаспознавание речи.
    \itemОбработка строк.
    \itemВекторная/матричная математика.
\end{itemize}

Для использования функций данной библиотеки необходимо в исходном коде подключить заголовочный файл IPP: 
\mint{c++}{#include<ipp.h>}

Рассмотрим пример программы которая вычисляет модуль синуса каждого элемента массива:

\begin{minted}{c++}
for (int i=0; i<N; i++) {
    array[i] = abs(sin(array[i]));
}
\end{minted}

Теперь воспользуемся функциями IPP, тогда наша программа будет выглядеть так:

\begin{minted}{c++}
ippsSin_64f_A21(array, array, N);
ippsAbs_64f_A21(array, array, N);
\end{minted}

Благодаря использованию данных функция, программа стала компактнее и быстрее.

Более подробно об использовании функций IPP можно узнать из официальной документации~\cite{IppPrimitives}.
% Все об использовании функций IPP можно узнать из официальной документации~\cite{IppPrimitives}.
% \url{https://software.intel.com/content/www/us/en/develop/documentation/ipp-dev-reference/top.html}.
