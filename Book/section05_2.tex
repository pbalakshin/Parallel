\subsection{Состав отчета}
\begin{enumerate}
    \itemТитульный лист с названием вуза, ФИО студента и названием работы.
    \itemСодержание отчета (с указанием номера страниц и т.п.).
    \itemОписание решаемой задачи (взять из п.~5.5).
    \itemКраткая характеристика использованного для проведения экспериментов процессора, операционной системы и компилятора GCC (официальное название, номер версии/модели, разрядность, число ядер, ёмкость ОЗУ, размер кэша и т.п.).
    \itemПолный текст программы lab1.c в виде отдельного файла.
    \itemТаблицы значений и графики функций seq(N), par-K(N) с указанием времени выполнения и величины параллельного ускорения. Предпочтительно использовать столбчатые гистограммы, показывающие зависимости времени или ускорения от размера массива.
    \itemПодробные выводы с анализом приведённых графиков и полученных результатов.
    \itemОтчёт предоставляется в бумажном или электронном виде вместе с полным текстом программы. По требованию преподавателя нужно быть готовыми скомпилировать и запустить этот файл на компьютере в учебной аудитории (или своём ноутбуке).
\end{enumerate}
