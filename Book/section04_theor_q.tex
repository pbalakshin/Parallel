\section{Вопросы для самоконтроля усвоенных знаний}

Следующие вопросы позволят оценить степень усвоения знаний с учетом применения лекционного материала и рекомендованной литературы.

\begin{enumerate}
    %%% Лекция 1
    %\item What is the main difference between SMP and MPP systems?
    \item В чем заключается основное различие между SMP- и MPP-системами?
    %\item 4.	During the existence of computer technology, the speed of triggering of elements has increased 10^6 times, and the speed of calculations has increased 10^9 times. How this can happen (means why speed increase in 10^9, not 10^6)?
    \item За время существования вычислительной техники скорость срабатывания элементов увеличилась в $10^6$ раз, а скорость вычислений -- в $10^9$ раз. Как это могло произойти (имеется в виду, почему скорость увеличилась в $10^9$, а не в $10^6$)?
    %\item Assume that you have a program with parallel fraction k = 0.8 and number of processors p = 4. Based on Amdahl’s law calculate parallel speed up S(p) and parallel efficiency EA(p).
    \item Предположим, что имеется программа с параллельной долей $k = 0,8$ и числом процессоров $p = 4$. На основе закона Амдала рассчитайте параллельное ускорение $S(p)$ и параллельную эффективность $E_A(p)$.

    %%% Лекция 2
    %\item You have machine with 4 cores. Thread "A" is running on the 1st core. This thread creates child threads and fibers. Which cores can be used to run such child threads and fibers?
    \item Имеется вычислительная машина с четырьмя ядрами. На первом ядре запущен поток "A". Этот поток создает дочерние потоки и волокна. Какие ядра могут быть использованы для запуска таких дочерних потоков и волокон?
    %\item There is a code that you executed on 4 cores machine.
    % Program was interrupted during execution on thread #3. Which thread will be used to handle additional swap from interrupt_handler?
    \item Имеется код, который выполнялся на вычислительной машине с четырьмя ядрами.
    \begin{minted}{c++}
    __thread int t;
    void swap(int *x, int *y) {
        t = *x;
        *x = *y;
        // hardware interrupt
        *y = t;
    }
    void interrupt_handler() {
        int x = 1, y = 2;
        swap(&x, &y);
    }
    \end{minted}

    Программа была прервана во время выполнения на потоке №3. Какой поток будет использоваться для обработки дополнительного вызова функции swap() от обработчика прерываний?

    %\item How is a thread-safe function different from a reenterable function?
    \item Чем отличается потокобезопасная функция от реентерабельной?

    %\item Specify the main disadvantages of Hyper-Threading.
    \item Укажите основные недостатки использования Hyper-Threading.
    
    %%% Лекция 3
    %\item What is the main difference between instruction and data parallelism?
    \item В чем заключается основное различие между распараллеливанием по данным и распараллеливанием по задачам?

    %\itemFrom the False sharing point of view is it good or bad to have bigger cache memory size? Why?
    \item С точки зрения проблемы False sharing хорошо или плохо иметь больший размер кэш-памяти? Почему?
    
    %\item There is an information about some program:
    %1). Populate an array A (size = 10 elements) with the first 10 elements of Fibonacci sequence. 
    %2). Populate an array B (same size) with elements from array A, but multiply even elements of array A by 2. Array A will not be changed.
    %3). Summarize elements of array A -> variable C.
    %4). Summarize elements of array B -> variable D.
    %5). Compare variables C and D. 

    %Which parallelization can be used (data, instructions, information flows) and why?
    \item Имеется информация о некоторой программе:
    \begin{enumerate}
        \item Заполнить массив A (размер = 10 элементов) первыми 10 элементами последовательности Фибоначчи. 
        \item Заполнить массив B (того же размера) элементами из массива A, но четные элементы массива A умножить на 2. Массив A не изменяется.
        \item Просуммировать элементы массива A и записать в переменную C.
        \item Просуммировать элементы массива B и записать в переменную D.
        \item Сравнить переменные C и D. 
    \end{enumerate}
    
    Какое распараллеливание можно использовать (по данным, по задачам, по информационным потокам) и почему?


    %%% Лекция 4
    %\item Modify the following sequential program by adding openmp constructions in order to be parallelized on proper amount of threads:

    \item Используя информацию о программе из предыдущего вопроса, модифицируйте следующую последовательную программу, добавив конструкции OpenMP, чтобы распараллелить ее на соответствующее число потоков:
    \begin{minted}{c++}
    populate_array(A);
    populate_array(B);
    sum_elements_array(A, C);
    sum_elements_array(B, D);
    compare(C, D)
    \end{minted}
   
    
    %\item What is the main difference between forward and backward compatibility?
    \item В чем заключается основное различие между прямой и обратной совместимостью?

    
    %%% Лекция 5
    %\item The following program is used to calculate number s as the sum of integers from 0 to 49. The program is compiled using OpenMP technology and runs on a dual-core processor. Why is the program giving the wrong result?
    \item Следующая программа используется для вычисления числа $s$ как суммы целых чисел от 0 до 50. Программа скомпилирована с использованием технологии OpenMP и выполняется на двухъядерном процессоре.

    \begin{minted}{c++}
    #include <stdio.h>
    #include <omp.h>
    void main(int argc, char* argv[]) {
        int s = 0, i = 0;
        #pragma omp parallel for
        for (int i = 0; i < 49; i++) s += i;
        printf("s = %d\n", s);
    }
    \end{minted}
    
    
    Почему программа выдает неверный результат?

    %\item How can you change the program from the previous question to give the correct result and use OpenMP's paralleling capabilities to find the sum? Provide all possible solutions.
    \item Как можно изменить программу из предыдущего вопроса, чтобы получить правильный результат и использовать возможности распараллеливания OpenMP для нахождения суммы? Приведите все возможные решения.
    
    %%% Лекция 6
    % \item Name the main differences between the master directive and the single directive
    %\item Перечислите основые отличие между директивой <<omp master>> и <<omp single>> в OpenMP.

    % \item Why nested style is faster than cascade one?
    \item Почему при использовании большого числа циклов в OpenMP время выполнения программы быстрее при использовании вложенного стиля?

    % \item What value n will be printed on the console?
    % \item What happens if firstprivate will be used instead of lastprivate?
    % \item What happens if private will be used instead of lastprivate?
    \item На основе следующего кода ответьте на несколько вопросов:
    \begin{minted}{c++}
    int n = 0;
    printf("Значение n в начале: %d\n", n);
    #pragma omp parallel for lastprivate(n)
    for (int i = 0; i < 6; ++i) {
        n = i * 10;
    }
    printf("Значение n в конце: %d\n", n);
    \end{minted}

    \begin{enumerate}[label*=\arabic*.]
        \item Какое значение $n$ будет выведено в конце?
        \item Как изменится значение $n$ в конце, если <<lastprivate>> заменить на <<firstprivate>>?
        \item Как изменится значение $n$ в конце, если <<lastprivate>> заменить на <<private>>?
    \end{enumerate}
    
    %%% Лекция 7
    % \item How many threads can be involved in deadlock? Explain your answer
     \item Какое число потоков необходимо для возможного появления взаимной блокировки? Объясните ответ.

     % \item What does transactional memory mean? Give examples of usage.
     \item Что такое программная транзакционная память? Приведите пример использования.

     % \item With the help of what function in POSIX THREAD the thread pool is merged (when one thread is waiting for another thread to finish).
     \item С помощью какой функции в POSIX Threads происходит слияние пула потоков (когда один поток ожидает завершения другого потока).
     
    
    %%% Лекция 8
    % \item You have an array of 317 element. Local work size = 12. Specify work_group_id and local_id for element 134
    \item Имеется массив из 317 элементов, который обрабатывается с помощью OpenCL, значение local\_work\_size = 12. Укажите work\_group\_id and local\_id для элемента с индексом 134.
    
    % \item Specify OpenCL memory types. How do they relate to each other?
    \item Перечислите типы памяти, существующие в OpenCL. Есть ли особенности взаимодействия разных типов памяти?
\end{enumerate}